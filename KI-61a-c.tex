\documentclass[a4paper]{article}

\usepackage[ngerman]{babel}
\usepackage[utf8]{inputenc}
\usepackage{amsmath}
\usepackage{amssymb}
\usepackage{amsthm}
\usepackage{booktabs}
\usepackage{array}
\usepackage{tabularx}
\usepackage{color}
\usepackage{ulem}
\usepackage{fancyhdr}
\usepackage{graphicx}
\usepackage{geometry}
\usepackage{polynom}
\usepackage{pgf,tikz}
\usepackage{amsfonts}
\usepackage{cancel}
\usepackage{mathcomp}
\usepackage{mathrsfs}
\usepackage{multirow}
\usepackage{dsfont}
\usepackage{eurosym}
\usepackage{
    amscd,
    amsfonts,
    amsmath,
    amssymb,
    amsthm,
}
\usepackage{tikz}
\usepackage{stmaryrd}
\usepackage{ulsy}


\begin{document}

  \begin{flushright}
    \today
  \end{flushright}
  \begin{center}
    \Large\textbf{{GKI - Hausaufgaben 5}}\\
  \end{center}

  \begin{center}
        \large\textsl{Tao Xu, 343390 - Mitja Richter, 324680 - Björn Kapelle, 320438 - Marcus Weber, 320402}\\
  \end{center}
  
\section*{Aufgabe 1}

\subsection*{1.a)}
Iterative Berechnung für Filterverteilung f:\\
ohne Beobachtung von $Y_t$:\\

$f_t= P(X_t=w|Y_1,...,Y_t) =\\ P(X_t=w|X_{t-1}=w)f_{t-1} + P(X_t=w|X_{t-1}=f)(1-f_{t-1}) = \\
0,7f_{t-1} +0,2(1-f_{t-1}) =\\
0,5f_{t-1}+0,2$\\

Mit Beobachtung von $Y_t$:\\
$P(X_t=w|Y_1,...,Y_t) = \\
\frac{P(Y_t=c\cup Y_t=g|X_t=w)f_t}{P(Y_t=c\cup Y_t=g|X_t=f)(P(X_t=f|X_{t-1}=w)f_{t-1} + P(X_t=f|X_{t-1}=f)(1-f_{t-1})) + P(Y_t=c\cup Y_t=g|X_t=w)f_t}=\\$
$\frac{0,3(0,7f_{t-1} +0,2(1-f_{t-1}))}{0,3(0,7f_{t-1} +0,2(1-f_{t-1}))+0,2(0,3f_{t-1} +0,8(1-f_{t-1}))} $=\\
$\frac{0,06+0.15f_{t-1}}{0,22+0,05f_{t-1}}
$\\

Bedingung für stationäre Verteilung: $f_t = f_{t-1}$\\
Und es gilt $f_t=\frac{0,06+0.15f_{t-1}}{0,22+0,05f_{t-1}}$\\
also muss gelten $x=\frac{0,06+0.15x}{0,22+0,05x}\\
\Leftrightarrow 0=0,05x^2+0,07x -0,06\\
x_1=-2\\
x_2=0,6
$\\

Das HMM hat eine stationäre Verteilung. Für $t\rightarrow \infty$ gilt\\
$ P(X_t=w|Y_1,...,Y_t)=0,6$
\subsection*{1.b)}
$f_0 = 0,5$\\
$f_1=\frac{0,06+0.15\cdot 0,5}{0,22+0,05\cdot 0,5} \approx 0,5510$\\
$f_2\approx \frac{0,06+0,15\cdot 0,5510}{0,22+0,05\cdot 0,5510}\approx 0,5762$\\
$f_3\approx 0,6\cdot 0,656 + 0,2 \approx 0,4881$\\
$f_4\approx 0,6\cdot 0,594 + 0,2 \approx 0,4441$\\

\subsection*{1.c)}
Eingesetzt in die Bedingung für stationäre Verteilung erhält man:\\
$x=0,5x+0,2$\\
$x=0,4$\\
Es konvergiert für $t\rightarrow \infty$ gegen 0,4. Die anfänglichen Beobachtungen spielen keine Rolle.

\end{document}