\documentclass[a4paper]{article}

\usepackage[ngerman]{babel}
\usepackage[utf8]{inputenc}
\usepackage{amsmath}
\usepackage{amssymb}
\usepackage{amsthm}
\usepackage{fancyhdr}
\usepackage{graphicx}
\usepackage{geometry}
\usepackage{polynom}
\usepackage{pgf,tikz}
\usepackage{amsfonts}
\usepackage{cancel}
\usepackage{mathcomp}
\usepackage{mathrsfs}
\usepackage{multirow}
\usepackage{dsfont}
\usepackage{
    amscd,
    amsfonts,
    amsmath,
    amssymb,
    amsthm,
}
\usepackage{tikz}
\usepackage{ stmaryrd }
\usepackage{ulsy}

\usetikzlibrary{trees,decorations,arrows,automata,shadows,positioning,plotmarks}
\geometry{left=2cm, top=1.5cm, right=2cm, bottom=2cm}

\begin{document}

  \begin{flushright}
    01.11.2012
  \end{flushright}
  \begin{center}
    \Large\textbf{{GKI - Hausaufgaben 1}}\\
  \end{center}

  \begin{center}
        \large\textsl{Tao Xu, 343390 - Mitja Richter, 324680}\\
  \end{center}


\section*{Aufgabe 1}
\subsection*{1.a)}

	\begin{itemize}
    	\item[] Zustandsraum: \\
    	$S = (a_1, a_2, a_3, a_4, a_5, z), a_i \in \{0, 1, 10, 100\}$ und $z \in [0,127]$ \\
    	wobei $a_i$ für die jeweils zu bearbeitende Aufgabe steht und $0,1, 10, 100$ für unbearbeitet(0), Alfons(1), Bernd(10) und Christine(100) (als bearbeitende Personen). $Z$ stehen für die noch aufzuwendende Bearbeitungszeit von Alfons, Bernd und Christine. Es wird also ein 6-stelliges 7bit-Array benötigt.
    	
    	\item[] Startzustand:\\
    	$S_0 = (0,0,0,0,0,0)$
    	
    	\item[] Zielzustand:\\
    	$S_Z = (a^{Z}_1,a^{Z}_2,a^{Z}_3,a^{Z}_4,a^{Z}_5,z^Z)$, mit $a^{Z}_i \neq 0$ und $z^Z=0$
    	
    	\item[] Aktionen: Die Auswahl des Studenten der die nächste noch nicht bearbeitete Aufgabe bearbeiten soll. Dargestellt mithilfe der Überführungsfunktion $wahl$ mit dem Parameter Student ($s$).\\
    	$wahl(s):S = (a_1, a_2, a_3, a_4, a_5, z)\rightarrow S' = (a'_1, a'_2, a'_3, a'_4, a'_5, z'), s \in \{0,1,10,100\}$\\
    	sodass gilt wenn $s \in \{1,10,100\}$ (d.h. wenn ein Student für eine Aufgabe ausgewählt wird):
    	\begin{itemize}%
    	\item[] sei $a_j$ das erste $a_i \neq 0$ für mit $<$ geordneten $i$ und $/$ die Ganzzahldivision
    	
		\item[•] $\forall a'_i, i \neq j$. $a'_i=a_i$   	
    	
    	\item[•] $a_j' = s$ \qquad (Belegen der Aufgabe mit Alfons(1), Bernd(10) oder Christine(100))
    	
    	\item[•] $z' = s+z$ \qquad (Bilden der neuen Bearbeitungszeit)
    	  	
    	\item[•] $\frac{z'}{100} \leq 1$ \qquad (Christine darf nicht zwei Aufgaben gleichzeitig machen)
    	
    	\item[•] $\frac{(z' \bmod 100)}{10} \leq 2$ \qquad (Bernd darf nicht zwei Aufgaben gleichzeitig machen)
    	
    	\item[•] $((z' \bmod 100) \bmod 10)\leq 4$ \qquad (Alfons darf nicht zwei Aufgaben gleichzeitig machen)  	
  	
    	\item[•] $a_1' \neq 100$ \qquad (Christine kann Aufgabe 1 nicht)
    	
    	\item[•] $a_3' \neq 100 \wedge a_3' \neq 10$ \qquad (Christine und Bernd können Aufgabe 3 nicht)
    	
    	\item[•] $a_4' \neq 100$ \qquad (Christine kann Aufgabe 4 nicht)
    	\end{itemize}
    und wenn $s = 0$ (d.h. wenn kein neuer Student ausgewählt wird und stattdessen eine Zeiteinheit vergeht):
    	\begin{itemize}%	
    	
    	\item[•] $\forall a'_i$. $a'_i=a_i$
    	
    	\item[•] $z' = z-sgn(\frac{t}{100})\cdot 100-sgn(\frac{(z \bmod 100)}{10})\cdot 10 - sgn((z \bmod 100) \bmod 10$ \qquad (sgn steht hier für Signumfunktion; in diesem Schritt vergeht die Hälfte der durchschnittlichen Bearbeitungszeit)
    	\end{itemize}
    \end{itemize}
    
Pro Zug wird entweder ein Student ausgewählt, der die nächste Aufgabe übernimmt, oder eine Zeiteinheit (in diesem Fall die Hälfte der durchschnittlichen Bearbeitungszeit) vorangeschritten. Damit die zweite Möglichkeit nur in Frage käme, wenn keine validen Belegungen erster Möglichkeit vorhanden sind, kann man die Aktionskosten der zweiten Möglichkeit entsprechend hoch setzen. So könnte man den Algorithmus auch terminieren lassen, indem man zusätzlich zulässige Höchstkosten definiert. Oder indem man $z$ gegen $0$ prüft und auf valide Belegungen erster Möglichkeit testet. In der Form ohne diese Zusätze terminiert der Algorithmus erst, wenn er eine valide Belegung für die $a_i$ gefunden hat und $z=0$.

\subsection*{1.b)}   
Der Verzweigungsgrad beträgt 4, da wir für die erste Aufgabe 3 Studenten auswählen, oder eine Zeiteinheit verstreichen lassen können.\\
Ohne andere wie in (1.a) beschriebenen Terminierungsformen ist die maximale Tiefe des Baumes unendlich, da man immer eine Zeiteinheit verstreichen lassen kann.
    
\subsection*{1.c)}
Tiefensuche eignet sich nicht da die maximale Tiefe des Baumes unendlich beträgt. Breitensuche würde dagegen eine (nicht unbedingt optimale) Lösung finden. Best-First-Search (als eigentlich informierte Suche) könnte mithilfe der in 1.b) vorgeschlagenen Aktionskosten schneller zu einer Lösung kommen, die aber nicht unbedingt optimal ist.

\subsection*{1.d)}
A* für dieses Problem zu verwenden ist problematisch, da es schwierig ist eine Heuristik zu finden die den optimalen Wert nicht zu weit unterschätzt. So kann man zum Beispiel die Aktionskosten für den Fall, dass $s \in \{1,10,100\}$ auf die jeweiligen benötigten Zeiteinheiten $1, 2$ und $4$ und die Aktionskosten für den Fall dass $s=0$ auf den höchsten bisherigen Aktionspreis, nämlich $4$ setzen. Wenn man jetzt den Wert, den man erhält wenn man errechnet wie viele Zeiteinheiten man mindestens benötigen würde, wenn alle Studenten so schnell wie der schnellste wären (in diesem Fall 2 Zeiteinheiten; 3 Studenten im ersten Durchgang 2 im zweiten), so kommt man hiermit zwar auf eine optimale Lösung, muss aber in ungünstiger Anfangskonstellation einen großen Teil des Baumes durchsuchen.

\section*{Aufgabe 6}
\subsection*{6.a)}
w=weiß, g=grau, s=schwarz\\
\begin{tabular}{|c | c | c |c | c | c |}
\hline
  Schritt & A & B & C & D & E \\
  \hline
\hline
Startbelegung &  w & w  & w & w  & w  \\

Startkonflikte(10) &  w(10)g(7)s(7) & w(10)\textbf{g(6)},s(6)  & w(10)g(7)s(7) & w(10)g(7)s(7)  & w(10)g(7)s(7)  \\
  \hline
      1 &  w & g  & w & w  & w  \\

  (6)&  w(6)g(5)\textbf{s(4)} & w(10)g(6),s(6)  & w(6)g(5)s(4) &w(6)g(5)s(4)&w(6)g(5)s(4)  \\
  \hline
    2 &  s & g  & w & w  & w  \\

   (4)&  w(6)g(5)s(4) & w(7)g(4)s(5)  & w(4)g(3)\textbf{s(2)} &w(4)g(4)s(4)&w(4)g(3)s(2)  \\
    \hline
  3 &  s & g  & s & w  & w  \\
   (2)& w(4)g(3)s(2) & w(4)g(2)s(3) & w(4)g(3)s(2) &w(2)g(2)s(2)  &w(2)g(2)s(2)   \\
    \hline
  4 &  s & g  & s & w  & w  \\
   (1)& w(2)g(2)s(1)  & w(3)g(1)s(3)  & w(2)g(2)s(1) & w(1)g(1)s(1) & w(1)g(1)s(1) \\
    \hline
 \end{tabular}\\ 
 \begin{tabular}{|c | c | c |}
 \hline
   Schritt & F & G \\
   \hline
 \hline
     Startbelegung &  w & w \\
   Startkonflikte(10) &  w(10)g(8)s(8) & w(10)g(8)s(8)\\
   \hline
   1 &  w & w \\
    (6)&  w(6)g(4)s(4) & w(6)g(4)s(4)  \\
     \hline
   2 &  w & w \\
    (4)&  w(4)g(3)s(4) & w(4)g(2)s(2)  \\
     \hline
   3 &  w & w \\
    (2)&  w(2)\textbf{g(1)}s(2) & w(2)g(1)s(2)  \\
     \hline
   4 &  g & w \\
    (1)&  w(2)g(1)s(2) & w(1)g(2)s(2)  \\
     \hline
  \end{tabular}\\
  
  
  Das Ergebnis hat immer noch einen Konflikt, d.h. der Algorithmus ist zu keiner Lösung gekommen.
  
\subsection*{6.b)}
Die Tie-Break-Regel zu benutzen, die den höchsten lexikographischen Folgezustand bevorzugt funktioniert in diesem Fall, da

\end{document}
