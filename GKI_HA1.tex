\documentclass[a4paper]{article}

\usepackage[ngerman]{babel}
\usepackage[utf8]{inputenc}
\usepackage{amsmath}
\usepackage{amssymb}
\usepackage{amsthm}
\usepackage{fancyhdr}
\usepackage{graphicx}
\usepackage{geometry}
\usepackage{polynom}
\usepackage{pgf,tikz}
\usepackage{amsfonts}
\usepackage{cancel}
\usepackage{mathcomp}
\usepackage{mathrsfs}
\usepackage{multirow}
\usepackage{dsfont}
\usepackage{
    amscd,
    amsfonts,
    amsmath,
    amssymb,
    amsthm,
}
\usepackage{tikz}
\usepackage{ stmaryrd }
\usepackage{ulsy}

\newcommand{\IR}{\mathbb{R}}

\usetikzlibrary{trees,decorations,arrows,automata,shadows,positioning,plotmarks}
\geometry{left=2cm, top=1.5cm, right=2cm, bottom=2cm}

\parindent0pt

\begin{document}

  \begin{flushright}
    \today
  \end{flushright}
  \begin{center}
    \Large\textbf{{GKI - Hausaufgaben 1}}\\
  \end{center}

  \begin{center}
        \large\textsl{Tao Xu, 343390 - Mitja Richter, 324680 - Björn Kapelle, 320438 - Marcus Weber, 320402}\\
  \end{center}


\section*{Aufgabe 1}
\subsection*{1.a)}

	\begin{itemize}
    	\item[] Zustandsraum: \\
    	$S = (a_1, a_2, a_3, a_4, a_5, z), a_i \in \{0, 1, 10, 100\}$ und $z \in [0,127]$ \\
    	wobei $a_i$ für die jeweils zu bearbeitende Aufgabe steht und $0,1, 10, 100$ für unbearbeitet(0), Alfons(1), Bernd(10) und Christine(100) (als bearbeitende Personen). $Z$ stehen für die noch aufzuwendende Bearbeitungszeit von Alfons, Bernd und Christine. Es wird also ein 6-stelliges 7bit-Array benötigt.
    	
    	\item[] Startzustand:\\
    	$S_0 = (0,0,0,0,0,0)$
    	
    	\item[] Zielzustand:\\
    	$S_Z = (a^{Z}_1,a^{Z}_2,a^{Z}_3,a^{Z}_4,a^{Z}_5,z^Z)$, mit $a^{Z}_i \neq 0$ und $z^Z=0$
    	
    	\item[] Aktionen: Die Auswahl des Studenten der die nächste noch nicht bearbeitete Aufgabe bearbeiten soll. Dargestellt mithilfe der Überführungsfunktion $wahl$ mit dem Parameter Student ($s$).\\
    	$wahl(s):S = (a_1, a_2, a_3, a_4, a_5, z)\rightarrow S' = (a'_1, a'_2, a'_3, a'_4, a'_5, z'), s \in \{0,1,10,100\}$\\
    	sodass gilt wenn $s \in \{1,10,100\}$ (d.h. wenn ein Student für eine Aufgabe ausgewählt wird):
    	\begin{itemize}%
    	\item[] sei $a_j$ das erste $a_i \neq 0$ für mit $<$ geordneten $i$ und $/$ die Ganzzahldivision
    	
		\item[•] $\forall a'_i, i \neq j$. $a'_i=a_i$   	
    	
    	\item[•] $a_j' = s$ \qquad (Belegen der Aufgabe mit Alfons(1), Bernd(10) oder Christine(100))
    	
    	\item[•] $z' = s+z$ \qquad (Bilden der neuen Bearbeitungszeit)
    	  	
    	\item[•] $\frac{z'}{100} \leq 1$ \qquad (Christine darf nicht zwei Aufgaben gleichzeitig machen)
    	
    	\item[•] $\frac{(z' \bmod 100)}{10} \leq 2$ \qquad (Bernd darf nicht zwei Aufgaben gleichzeitig machen)
    	
    	\item[•] $((z' \bmod 100) \bmod 10)\leq 4$ \qquad (Alfons darf nicht zwei Aufgaben gleichzeitig machen)  	
  	
    	\item[•] $a_1' \neq 100$ \qquad (Christine kann Aufgabe 1 nicht)
    	
    	\item[•] $a_3' \neq 100 \wedge a_3' \neq 10$ \qquad (Christine und Bernd können Aufgabe 3 nicht)
    	
    	\item[•] $a_4' \neq 100$ \qquad (Christine kann Aufgabe 4 nicht)
    	\end{itemize}
    und wenn $s = 0$ (d.h. wenn kein neuer Student ausgewählt wird und stattdessen eine Zeiteinheit vergeht):
    	\begin{itemize}%	
    	
    	\item[•] $\forall a'_i$. $a'_i=a_i$
    	
    	\item[•] $z' = z-sgn(\frac{t}{100})\cdot 100-sgn(\frac{(z \bmod 100)}{10})\cdot 10 - sgn((z \bmod 100) \bmod 10$ \qquad (sgn steht hier für Signumfunktion; in diesem Schritt vergeht die Hälfte der durchschnittlichen Bearbeitungszeit)
    	\end{itemize}
    \end{itemize}
    
Pro Zug wird entweder ein Student ausgewählt, der die nächste Aufgabe übernimmt, oder eine Zeiteinheit (in diesem Fall die Hälfte der durchschnittlichen Bearbeitungszeit) vorangeschritten. Damit die zweite Möglichkeit nur in Frage käme, wenn keine validen Belegungen erster Möglichkeit vorhanden sind, kann man die Aktionskosten der zweiten Möglichkeit entsprechend hoch setzen. So könnte man den Algorithmus auch terminieren lassen, indem man zusätzlich zulässige Höchstkosten definiert. Oder indem man $z$ gegen $0$ prüft und auf valide Belegungen erster Möglichkeit testet. In der Form ohne diese Zusätze terminiert der Algorithmus erst, wenn er eine valide Belegung für die $a_i$ gefunden hat und $z=0$.

\subsection*{1.b)}   
Der Verzweigungsgrad beträgt 4, da wir für die erste Aufgabe 3 Studenten auswählen, oder eine Zeiteinheit verstreichen lassen können.\\
Ohne andere wie in (1.a) beschriebenen Terminierungsformen ist die maximale Tiefe des Baumes unendlich, da man immer eine Zeiteinheit verstreichen lassen kann.
    
\subsection*{1.c)}
Tiefensuche eignet sich nicht da die maximale Tiefe des Baumes unendlich beträgt. Breitensuche würde dagegen eine (nicht unbedingt optimale) Lösung finden. Best-First-Search (als eigentlich informierte Suche) könnte mithilfe der in 1.b) vorgeschlagenen Aktionskosten schneller zu einer Lösung kommen, die aber nicht unbedingt optimal ist.

\subsection*{1.d)}
A* für dieses Problem zu verwenden ist problematisch, da es schwierig ist eine Heuristik zu finden die den optimalen Wert nicht zu weit unterschätzt. So kann man zum Beispiel die Aktionskosten für den Fall, dass $s \in \{1,10,100\}$ auf die jeweiligen benötigten Zeiteinheiten $1, 2$ und $4$ und die Aktionskosten für den Fall dass $s=0$ auf den höchsten bisherigen Aktionspreis, nämlich $4$ setzen. Wenn man jetzt den Wert, den man erhält wenn man errechnet wie viele Zeiteinheiten man mindestens benötigen würde, wenn alle Studenten so schnell wie der schnellste wären (in diesem Fall 2 Zeiteinheiten; 3 Studenten im ersten Durchgang 2 im zweiten), so kommt man hiermit zwar auf eine optimale Lösung, muss aber in ungünstiger Anfangskonstellation einen großen Teil des Baumes durchsuchen.

\section*{Aufgabe 2}
\subsection*{2.a)}
F\"ur die Formulierung eines Suchproblems brauchen wir einen Startzustand, einen Zielzustand, Aktionen und Aktionskosten. Unsere Zustände sind Vektoren der Form $(A_1,A_2,B_1,B_2,B_3,B_4,C_1,C_2,D_1,D_2)$ wodurch gekennzeichnet werden kann welcher Waggon auf welchem Stellplatz ist, $1,2,3,4$ f\"ur die 4 Waggons und 0 bedeutet, dass der Stellplatz aktuell leer ist. Der Startzustand ist $(0,0,2,4,3,1,0,0,0,0)$. Der Zielzustand ist $(0,0,1,2,3,4,0,0,0,0)$. Die Aktionen, die wir durchf\"uhren k\"onnen sind abh\"angig von dem aktuellen Zustand in dem wir uns befinden. Die Aktionen haben aber gemeinsam, dass sie jeweils die Bewegung eines Waggons auf ein anderes Gleisst\"uck beschreiben.
Wir unterscheiden die Aktionen Move(X,Y) mit $X,Y \in \{A,B,C,D\}$ und $X \not= Y$. Mit den verschiedenen Kombinationsmöglichkeiten erhalten wir 12 verschiedene Aktionen. Dabei stellen wir allerdings einige Vorbedingungen damit eine Aktion auch durchf\"uhrbar ist. So muss mindestens ein $X_i \not= 0, i \in \{1,2\}$, falls $X$ gleich $A,B$ oder $C$ ist, und falls $X$ gleich $B$ soll $i \in \{1,2,3,4\}$ sein und mindestens ein $Y_i = 0$, wobei $i$ wie vorher.
Eine weitere Voraussetzung ist, dass die Waggons immer so weit in den Gleisst\"ucken durchrutschen wie m\"oglich, d.h. z.B. ist der Zustand $(0,1,0,2,3,4,0,0,0,0)$ nicht erlaubt.
Die Aktionskosten belaufen sich bei allen Aktionen auf 1, da wir m\"oglichst wenig Waggons verschieben wollen und jede Verschiebung gleich aufwendig ist.

\subsection*{2.b)}

\begin{figure}[h]
\centering
\includegraphics[width=0.75\columnwidth]{aufgabe2b}
\end{figure}

\subsection*{2.c)}
Der Verweigungsgrad ist die maximale Anzahl an Nachfolgerknoten f\"ur irgendeinen Knoten in unserem Suchbaum. Die Nachfolgeknoten entsprechen aber den Folgezust\"anden. Die maximale Anzahl an Folgezust\"anden ist 12, dies erhalten wir, wenn in jedem Gleisst\"uck ein Waggon liegt. Dann ist n\"amlich jede Aktion durchf\"uhrbar. Somit ist der Verzweigungsgrad 12.
Die Tiefe des Baumes ist nicht ohne weiteres zu benennen, aber wir kennen mit Gewissheit eine obere Schranke und zwar gilt: Tiefe des Suchbaums $\le \binom{10}{4} 4! = 5040$. Dies ergibt sich wenn wir alle erdenkbaren Zust\"ande (egal ob f\"ur uns zul\"assig oder nicht) nacheinander erhalten würden. Dabei beschreibt $\tbinom{10}{4}$ die Anzahl der verschiedenen besetzten Positionen auf allen Gleisst\"ucken und $4!$ beschreibt die unterschiedliche Positionierung der 4 Waggons auf den Positionen. Da wir Zykel nicht zulassen wollen, erreichen wir auch keine gr\"o{\ss}ere Tiefe.

\subsection*{2.d)}
Je nachdem auf was man Wert legt, ist entweder die Tiefensuche oder die Breitensuche besser. Der zeitliche Aufwand ist bei der Breitensuche besser, da diese $b^d$, wobei $b$ dem Verzweigungsgrad entspricht und $d$ der Tiefe der optimalen L\"osung entspricht. Dabei ist zu erw\"ahnen, dass die Tiefe der optimalen L\"osung die minimale Tiefe des Zielzustands ist. Diese ist mit Sicherheit kein Blatt, da es 3 Vorg\"angerzust\"ande gibt, die zum Zielzustand f\"uhren k\"onnen und alle drei nicht vorher besucht werden bei einer optimalen L\"osung.
Dies ist entscheidend, da der zeitliche Aufwand in der Tiefensuche $b^m$ ist, wobei $m$ der maximalen Baumtiefe entspricht. 
Der Speicheraufwand ist hingegen bei der Tiefensuche besser als bei der Breitensuche, da dieser bei der Breitensuche $b^d$ entspricht und bei Tiefensuche lediglich $b \cdot\ m$. Der exponentielle Anstieg bei der Breitensuche hat einen deutlich st\"arkeren Effekt als der lineare Anstieg bei der Tiefensuche,da $d$ mindestens 8 ist (weil jeder Waggons mindestens 2 mal bewegt werden muss, sonst kann der Zielzustand nicht erreicht werden). $b$ ist gleich 12 und $m$ ist maximal 5040.
Es gilt, aber dass $12^8 > 12 \cdot\ 5040$. Somit hat man auch einen Zahlenbeleg f\"ur unsere Behauptung.
Daher liegt es in den Augen des Betrachters welches Verfahren besser geeignet, je nach Interessenlage.

\section*{Aufgabe 3}
\subsection*{3.a)}
Die Iterative Tiefensuche startet bei einem Wurzelknoten und führt dort im ersten Schritt eine limitierte Tiefensuche mit minimaler Suchtiefe. In jedem Iterationsschritt wird nur erneut eine limitierte Tiefensuche durchgeführt, während die Suchtiefe um 1 erhöht wird. Abgebrochen wird, sobald eine konsistente Belegung gefunden wird, oder die maximale Suchtiefe erreicht ist. Die Performance des Algorithmus ist ähnlich der normalen Tiefensuche, allerdings ist er durch die Iteration die Optimalität betreffend auch ähnlich gut wie die Breitensuche.
Da jede Iteration teilweise gleiche Knoten aufgespannt werden müssen wirkt sich das (geringfügig) negativ auf die Laufzeit aus. Kombiniert man die Iterative Tiefensuche mit dem Minmax-Algorithmus, kommt der positive Effekt hinzu, dass man zu jedem Zeitpunkt eine Lösung abrufen kann.

\subsection*{3.b)}
Die Formel für alle generierten Knoten der Iterativen Tiefensuche ist:\\
$N_{ITS}=\sum\limits_{i=0}^{d} (d+1-i)b^i$, mit d = Tiefe und b=Verzweigung\\
Einsetzen ergibt:\\
$N_{ITS}=\sum\limits_{i=0}^{5^5} (5^5+1-i)35^i \approx 1,08 \cdot 10^{193}$\\

Für die limitierte Tiefensuche gilt:\\
$N_{LTS}=\sum\limits_{i=0}^{d} b^i$\\
Einsetzen ergibt:\\
$N_{LTS}=\sum\limits_{i=0}^{5^5} 35^i \approx 1,05 \cdot 10^{193}$\\

Das ergibt für den Overhead $\approx 2,8\%$

\subsection*{3.c)}
Je höher der Verzweigungsfaktor ist, desto kleiner ist der Overhead. Also eignet sich iterative Tiefensuche eher bei hohem Verzweigungsfaktor. Dieser Effekt kommt daher, dass bei Bäumen mit hohem Verzweigungsfaktor die Knoten in der maximale Tiefe (also den noch nicht erzeugten Knoten), einen sehr großen Teil des Baumes ausmachen. Je mehr Iteration man macht, desto größer wird dieser Anteil.

\section*{Aufgabe 4}

\subsection*{4.a)}
\begin{tabular}{|c | l | l |l |}
\hline
  Schritt & Expandierter Knoten & Queue & Anmerkungen\\
\hline
\hline
1 &  A(35)   & AC(30), AB(40)           & keine\\
\hline
2 &  AC(30)  & ACZ(35), AB(40), ACD(40) & ACA entfernt (Zykel)\\
  &          &                          & ACB(65) entfernt (DP)\\
\hline
3 &  ACZ(35) & -                        & Algorithmus terminiert\\
\hline
\end{tabular}\\ 

\subsection*{4.b)}
$A^\ast$ liefert nur dann mit Sicherheit die optimale L\"osung, wenn die zugrundeliegende Heuristik zul\"assig ist. In unserem Fall ist die Heuristik nicht zul\"assig, denn sie \"ubersch\"atzt die Kosten f\"ur A ($35>30$) und f\"ur B ($35>25$). Und in der Tat erhalten wir nicht das optimale Ergebnis.

\subsection*{4.c)}
Wir \"andern die Heuristik f\"ur A und f\"ur B, so dass $h(A)=25$ und $h(B)=20$ gilt. Dann folgt:\\
\\
\begin{tabular}{|c | l | l |l |}
\hline
  Schritt & Expandierter Knoten & Queue & Anmerkungen\\
\hline
\hline
1 & A(25)    & AB(25), AC(30)    & keine\\
\hline
2 & AB(25)   & ABC(25), ABD(30)  & ABA entfernt (Zykel)\\
  &          &                   & AC(30) entfernt (DP)\\
\hline
3 & ABC(25)  & ABCZ(30), ABD(30) & ABCA entfernt (Zykel)\\
  &          &                   & ABCB entfernt (Zykel)\\
  &          &                   & ABCD(35) entfernt (DP)\\
\hline
4 & ABCZ(30) & -                 & Algorithmus terminiert\\
\hline
\end{tabular}\\
\vspace{10pt}\\
Die Heuristik ist jetzt zul\"assig und $A^\ast$ liefert die optimale L\"osung.

\section*{Aufgabe 5}
Bei den Aufgaben a) bis c) ist die Zul\"assigkeit verschiedener Metriken als Heuristik zu zeigen. Eine Heuristik $h$ ist zul\"assig, wenn $0 = h(X) = h^\ast(X)$ f\"ur alle Knoten $X$, wobei $h^\ast$ die tats\"achlichen Kosten bezeichnet.

\subsection*{5.a)}
Wir betrachten als heuristische Funktion $h$ die euklidische Metrik. Die euklidische Metrik ist definiert als $d_{eukl}(x,y) = \sqrt{(x_2-x_1)^2+(y_2-y_1)^2} $, wobei $x:=(x_1,y_1)$ und $y:=(x_2,y_2)$. 

Da $(x_2-x_1)^2+(y_2-y_1)^2\geq0$, $\forall x_1,x_2,y_1,y_2 \in \IR$, ist die Wurzel dieses Terms definiert und da diese wiederum ein nichtnegatives Ergebnis ausgibt gilt: $0 \leq h(X)$ f\"ur alle m\"oglichen zweidimensionalen Koordinaten (L\"angen- und Breitengrade). Zudem beschreibt die euklidische Metrik die Luftlinienentfernung zwischen zwei Punkten, somit gilt auch: $h(X) \leq h^\ast(X) \Rightarrow$ die euklidische Metrik ist zul\"assig.

\subsection*{5.b)}
Wir betrachten als heuristische Funktion $h$ die Maximum-Metrik.

Da $|x_2-x_1| \geq 0$ und $|y_2-y_1| \geq 0$, $\forall x_1,x_2,y_1,y_2 \in \IR$, ist $\max(|x_2-x_1|,|y_2-y_1|)$ ebenfalls nichtnegativ. Somit gilt: $0 \leq h(X)$ f\"ur alle m\"oglichen zweidimensionalen Koordinaten (L\"angen- und Breitengrade).

Zudem gilt: $d_{max}(x,y) \leq d_{eukl}(x,y)$ (siehe d.) und somit gilt auch $h(X) \leq h^\ast(X) \Rightarrow$ die Maximum-Metrik ist zul\"assig.

\subsection*{5.c)}
Wir betrachten als heuristische Funktion $h$ die Manhattan-Metrik.

Diese Metrik ist nicht zul\"assig. Dies zeigen wir durch ein Beispiel an dem man erkennen kann, dass die Manhattan-Metrik f\"ur unser Problem die tats\"achlichen Kosten \"ubersch\"atzen kann. Dazu seien zwei St\"adte an den Koordinaten (0,0) und (1,1) die auf direktem Weg (Luftlinie) miteinander durch eine Stra{\ss}e verbunden sind.
Die Entfernung zwischen diesen St\"adten ist demnach $\sqrt{2}$, aber die Manhattan-Metrik gibt $|1 - 0| + |1 - 0| = 2$ als Ergebnis aus. Somit ist die Manhattan-Metrik f\"ur unser Beispiel nicht zul\"assig.

\subsection*{5.d)}
F\"ur gegebene reelle Wertepaare $x:=(x_1,y_1)$ und $y:=(x_2,y_2)$ gilt:
$d_{max}(x,y) \stackrel{1.}{\leq} d_{eukl}(x,y) \stackrel{2.}{\leq} d_{man}(x,y)$\\
\\
Beweis 1.: Ohne Beschr\"ankung der Allgemeinheit sei: $d_{max}(x,y) = \max(|x_2- x_1|,|y_2- y_1|) = |x_2- x_1|$.
(Der andere Fall w\"urde genau analog verlaufen)

$|x_2-x_1|=\sqrt{(|x_2-x_1|)^2}=\sqrt{(x_2-x_1)^2}\leq\sqrt{(x_2-x_1)^2+ (y_2-y_1)^2}$

Der vorletzte Schritt ergibt sich aus der Tatsache, dass $(x_2-x_1)^2\geq0$ und somit der
Betrag hinf\"allig ist. Der letzte Schritt ergibt sich aus der Monotonie der Wurzelfunktion
und der Tatsache, dass $(x_2- x_1)^2+(y_2- y_1)^2\geq(x_2- x_1)^2$.\\
\\
Beweis 2.: Seien $a,b \in \IR^+_0$. Dann gilt: $a^2+ b^2= a^2+ 2ab + b^2= (a + b)^2$ und somit $\sqrt{a^2+b^2}\leq\sqrt{(a + b)^2} \stackrel{a,b\geq0}{=}(a + b)$ (Es wurde wieder die Monotonie der Wurzelfunktion
benutzt).

Sei nun $a := |x_2- x_1| \geq 0$ und $b :=|y_2- y_1| \geq 0$ und damit folgt die Behauptung.

\subsection*{5.e)}
Die Manhattan-Metrik ist nicht die geeignetste, da sie nicht zul\"assig ist und da die euklidische Metrik die Maximum-Metrik dominiert, ist die euklidische Metrik f\"ur unser Problem die geeignetste, da ihre Werte n\"aher an den tats\"achlichen Kosten liegen.

\subsection*{5.f)}

\begin{figure}[h]
\centering
\includegraphics[width=0.75\columnwidth]{aufgabe5f}
\end{figure}


An diesem Beispiel kann man sehen, dass wenn man den Algorithmus durchf\"uhren m\"ochte, um einen k\"urzesten Weg von D nach A zu finden, man alle beiden Wege \"uberpr\"ufen wird. Allerdings w\"urde man dies nicht tun, wenn statt -6 als heuristischer Wert 0 eingesetzt werden w\"urde, da der Gesamtwert nicht mehr 4 sondern 10 w\"are und somit in Anbetracht der Gesamtkosten des Pfades A-C-D gar nicht gepr\"uft werden m\"usste. Bei gr\"o{\ss}eren Beispielen kann dies zu erheblichen Teilpfaden f\"uhren, die man weglassen k\"onnte und somit effektiver arbeiten.

\subsection*{5.g)}
Die Vollst\"andigkeit des Algorithmus ist weiterhin gegeben, da wenn ein Pfad nicht zu einem Zielknoten f\"ihrt, ein anderer gepr\"uft werden w\"urde und dies so lange bis mindestens ein Punkt erreicht wird an dem Start- und Zielknoten wie gew\"unscht sind. Sollte kein
derartiger Pfad existieren, kann der Algorithmus keinen Pfad finden und von daher kann man solche F\"alle vernachl\"assigen.

Die Optimalit\"at wird ebenfalls nicht beeintr\"achtigt, da die heuristische Werte nur Sch\"atzwerte sind, die ein schnelleres Finden des besten Weges erm\"oglichen sollen. Wenn wir negative heuristische Werte zulassen, finden wir den gew\"unschten Weg evtl. nicht schneller, aber solange wir Heuristiken benutzen, die die tats"achlichen Kosten untersch\"atzen, erreichen wir trotzdem die beste L\"osung.

Dies gilt, weil f\"ur die Gesamtkosten der Wege am Ende die heuristischen Werte keinen Einfluss haben. Die \"Uberpr\"ufung anderer - vielleicht besserer - Wege wird gew\"ahrleistet durch die Untersch\"atzung, denn somit wird kein potenzieller Weg durch die heuristischen Werte als "`nicht weiter zu \"uberpr\"ufen"' gekennzeichnet.

\section*{Aufgabe 6}
\subsection*{6.a)}
w=weiß, g=grau, s=schwarz\\
\begin{tabular}{|c | c | c |c | c | c |}
\hline
  Schritt & A & B & C & D & E \\
  \hline
\hline
Startbelegung &  w & w  & w & w  & w  \\

Startkonflikte(10) &  w(10)g(7)s(7) & w(10)\textbf{g(6)},s(6)  & w(10)g(7)s(7) & w(10)g(7)s(7)  & w(10)g(7)s(7)  \\
  \hline
      1 &  w & g  & w & w  & w  \\

  (6)&  w(6)g(5)\textbf{s(4)} & w(10)g(6),s(6)  & w(6)g(5)s(4) &w(6)g(5)s(4)&w(6)g(5)s(4)  \\
  \hline
    2 &  s & g  & w & w  & w  \\

   (4)&  w(6)g(5)s(4) & w(7)g(4)s(5)  & w(4)g(3)\textbf{s(2)} &w(4)g(4)s(4)&w(4)g(3)s(2)  \\
    \hline
  3 &  s & g  & s & w  & w  \\
   (2)& w(4)g(3)s(2) & w(4)g(2)s(3) & w(4)g(3)s(2) &w(2)g(2)s(2)  &w(2)g(2)s(2)   \\
    \hline
  4 &  s & g  & s & w  & w  \\
   (1)& w(2)g(2)s(1)  & w(3)g(1)s(3)  & w(2)g(2)s(1) & w(1)g(1)s(1) & w(1)g(1)s(1) \\
    \hline
 \end{tabular}\\ 
 \begin{tabular}{|c | c | c |}
 \hline
   Schritt & F & G \\
   \hline
 \hline
     Startbelegung &  w & w \\
   Startkonflikte(10) &  w(10)g(8)s(8) & w(10)g(8)s(8)\\
   \hline
   1 &  w & w \\
    (6)&  w(6)g(4)s(4) & w(6)g(4)s(4)  \\
     \hline
   2 &  w & w \\
    (4)&  w(4)g(3)s(4) & w(4)g(2)s(2)  \\
     \hline
   3 &  w & w \\
    (2)&  w(2)\textbf{g(1)}s(2) & w(2)g(1)s(2)  \\
     \hline
   4 &  g & w \\
    (1)&  w(2)g(1)s(2) & w(1)g(2)s(2)  \\
     \hline
  \end{tabular}\\
  
  
  Das Ergebnis hat immer noch einen Konflikt, d.h. der Algorithmus ist zu keiner Lösung gekommen.
  
\subsection*{6.b)}
Die Tie-Break-Regel muss dahingehend ge\"andert werden, dass sie den den höchsten lexikographischen Folgezustand bevorzugt.\\
\\
In 6.a) ist das Problem, dass sehr fr\"uh A und C belegt werden. Dann ist aber D durch A und B festgelegt und E ist durch B und C festgelegt. Die Constraintverletzung zwischen D und E kann dann nicht mehr in einem Schritt gel\"ost werden und die lokale Suche bleibt somit in einem lokalen Minimum stecken.

Die ge\"anderte Tie-Break-Regel belegt nun hingegen A und C sp\"ater und daf\"ur E fr\"uher. Somit kommt es zu keinem Konflikt zwischen D und E. Die sp\"atere Belegung von A und C f\"uhrt zu keinen Problemen, denn deren Konflikte mit F bzw. G k\"onnen immer in einem Schritt gel\"ost werden, da F und G nur zwei Constraints haben.

\end{document}
