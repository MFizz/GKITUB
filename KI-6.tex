\documentclass[a4paper]{article}

\usepackage[ngerman]{babel}
\usepackage[utf8]{inputenc}
\usepackage{amsmath}
\usepackage{amssymb}
\usepackage{amsthm}
\usepackage{booktabs}
\usepackage{array}
\usepackage{tabularx}
\usepackage{color}
\usepackage{ulem}
\usepackage{fancyhdr}
\usepackage{graphicx}
\usepackage{geometry}
\usepackage{polynom}
\usepackage{pgf,tikz}
\usepackage{amsfonts}
\usepackage{cancel}
\usepackage{mathcomp}
\usepackage{mathrsfs}
\usepackage{multirow}
\usepackage{dsfont}
\usepackage{eurosym}
\usepackage{
    amscd,
    amsfonts,
    amsmath,
    amssymb,
    amsthm,
}
\usepackage{tikz}
\usepackage{stmaryrd}
\usepackage{ulsy}


\begin{document}

  \begin{flushright}
    \today
  \end{flushright}
  \begin{center}
    \Large\textbf{{GKI - Hausaufgaben 5}}\\
  \end{center}

  \begin{center}
        \large\textsl{Tao Xu, 343390 - Mitja Richter, 324680 - Björn Kapelle, 320438 - Marcus Weber, 320402}\\
  \end{center}
  
\section*{Aufgabe 1}

\subsection*{1.a)}
Iterative Berechnung für Filterverteilung f:\\
ohne Beobachtung von $Y_t$:\\

$f_t= P(X_t=w|Y_1,...,Y_t) =\\ P(X_t=w|X_{t-1}=w)f_{t-1} + P(X_t=w|X_{t-1}=f)(1-f_{t-1}) = \\
0,8f_{t-1} +0,2(1-f_{t-1}) =\\
0,6f_{t-1}+0,2$\\

Mit Beobachtung von $Y_t$:\\
$P(X_t=w|Y_1,...,Y_t) = \\
\frac{P(Y_t=c\cup Y_t=g|X_t=w)f_t}{P(Y_t=c\cup Y_t=g|X_t=f)(P(X_t=f|X_{t-1}=w)f_{t-1} + P(X_t=f|X_{t-1}=f)(1-f_{t-1})) + P(Y_t=c\cup Y_t=g|X_t=w)f_t}=\\$
$\frac{0,3(0,8f_{t-1} +0,2(1-f_{t-1}))}{0,3(0,8f_{t-1} +0,2(1-f_{t-1}))+0,2(0,2f_{t-1} +0,8(1-f_{t-1}))} $=\\
$\frac{0,06+0.18f_{t-1}}{0,22+0,06f_{t-1}}
$\\

Bedingung für stationäre Verteilung: $f_t = f_{t-1}$\\
Und es gilt $f_t=\frac{0,06+0.18f_{t-1}}{0,22+0,06f_{t-1}}$\\
also muss gelten $x=\frac{0,06+0.18x}{0,22+0,06x}\\
\Leftrightarrow 0=p^2+\frac{2}{3}p-1\\
x_1=-1,3874\\
x_2=0,72076
$\\

Das HMM hat eine stationäre Verteilung. Für $t\rightarrow \infty$ gilt\\
$ P(X_t=w|Y_1,...,Y_t)=0,72076$
\subsection*{1.b)}
$f_0 = 0,5$\\
$f_1=\frac{0,06+0.18\cdot 0,5}{0,22+0,06\cdot 0.5} = 0,6$\\
$f_2=\frac{0,06+0.18\cdot 0,5}{0,22+0,06\cdot 0.5}$\approx 0,656\\
$f_3\approx 0,6\cdot 0,656 + 0,2 \approx 0,594\\
$f_4\approx 0,6\cdot 0,594 + 0,2 \approx 0,556\\

\subsection*{1.c)}
Eingesetzt in die Bedingung für stationäre Verteilung erhält man:\\
$x=0,6x+0,2$\\
$x=0,5$\\
Es konvergiert für $t\rightarrow \infty$ gegen 0,5. Die anfänglichen Beobachtungen spielen keine Rolle.

\subsection*{1.e)}
\begin{tabular}{c|cccc}
    & a 		& c 			& g 			& t 			\\
\hline
w	& 0,1		& 0,021 (w)		& 0,00441 (w)		& 0,0006174 (w)	\\
f 	& 0,15		& 0,024 (f)		& 0,00384 (f)		& 0,0009216 (f)	\\
\end{tabular}

Zur Erkl\"arung dieser Tabelle.
Wir beschreiben hier aus Gr\"unden der Einfachheit nur die ersten zwei Spalten.
Die erste Spalte ergibt sich aus: $\frac{P(Y_0=a|X_0=Zustand)}{2}$
F\"ur Zustand kann einerseits $w$ eingesetzt werden, das ergibt den Wert in der mit $w$ gekennzeichneten Zeile. \\
Analog f\"ur die Zeile mit $f$. Der Nenner ergibt sich aus der Gesamtanzahl der Zust\"ande, in diesem Fall gibt es die Zust\"ande $w$ und $f$ und somit insgesamt 2. \\
$Y_0=a$ ergibt sich aus der gegeben Sequenz.\\
\\
Zur Berechnung der zweiten Spalte (und der weiteren Spalten): \\
man sucht in der $w$-Zeile das Maximum von: $max_{i \in \{w,f\}} v_{i,1} \cdot P(X_1=w | X_0= i) \cdot P(Y_1=c | X_1 = w)$ \\
$v_{i,k}$ beschreibt den Wert in Zeile $i$ und Spalte $k$, im Fall $v_{w,1}$ also z.B. den Wert $0,1$. \\
Die Beschreibung (w) oder (f) dahinter beschreibt welchen Zustandl vorher gewesen sein muss. 
\\
Rechnungen f\"ur $w$-Zeile: $0,1 \cdot 0,7 \cdot 0,3 = 0,021$  \\
und $0,15 \cdot 0,2 \cdot 0,3 = 0,009$ \\
Das Maximum dieser beiden ist der Wert $0,021$ \\
Da dies im Fall $i = w$ war, muss vorher der Fall $w$ gewesen sein. \\
\\
F\"ur die $f$-Zeile analog: $max_{i \in \{w,f\}} v_{i,1} \cdot P(X_1=f | X_0= i) \cdot P(Y_1=c | X_1 = f)$ \\
Rechnungen f\"ur $f$-Zeile: $0,1 \cdot 0,3 \cdot 0,2 = 0,006$  \\
und $0,15 \cdot 0,8 \cdot 0,2 = 0,024$ \\
Das Maximum dieser beiden ist der Wert $0,021$ \\
Da dies im Fall $i = f$ war, muss vorher der Fall $f$ gewesen sein. \\
\\
Da in der letzten Spalte der h\"ochste Wert in der $f$-Zeile vorhanden ist, ergibt sich, dass $X_3 = f$ und von da aus r\"uckwirkend die Zustandsfolge $X_0 = f, X_1=f, X_2=f, X_3=f$ die wahrscheinlichste für die gegebene Sequenz ist.

\section*{Aufgabe 2}

\subsection*{2.a)}
\begin{figure}[!htbp]
\centering
g\includegraphics[width = 1\columnwidth]{blatt6aufgabe2a}
\end{figure}

\subsection*{2.b)}
\glqq He\grqq $\in$ \{Subjekt\} \\
\glqq shoots\grqq $\in$ \{Verb\} \\
\glqq the\grqq $\in$ \{Artikel\} \\
\glqq unwell\grqq $\in$ \{Adjektiv(1), Adjektiv(2)\} \\
\glqq well\grqq $\in$ \{Adjektiv(1), Nomen, Adverb, Adjektiv(2)\} \\
\glqq badly\grqq $\in$ \{Adverb\} \\
\pagebreak
Daraus folgt das folgende Bild.
\begin{figure}[!htbp]
\centering
\includegraphics[width = 1\columnwidth]{blatt6aufgabe2b}
\end{figure}

Die einzigen möglichen Folgen von Wortarten beschreiben Pfade von der Wurzel (Subjekt) zum Blatt (Satzende) bei dem jede Kante eine positive Wahrscheinlichkeit hat.
Es gibt nur einen Pfad, der das erf\"ullt: Subjekt-Verb-Artikel-Adjektiv(2)-Nomen-Adverb-Satzende. \\
Insbesondere gibt es damit eine M\"oglichkeit diesen Satz zu bilden. \\
Im weiteren verwenden wir Abkürzungen:  \\
Sub - Subjekt \\
Ver - Verb \\
Hil - Hilfsverb \\
Art - Artikel \\
Adj(2) - Adjektiv(2) \\
Nom - Nomen \\
Adv - Adverb \\
SE - Satzende \\
\\
$P($passende Wortartenfolge$)$
$= P(x_0=Sub, x_1=Ver, x_2= Art, x_3=Adj(2), x_4=Nom, x_5=Adv, x_6=SE)$ \\
$= P(Ver | Sub)  P(Art | Ver) P(Adj(2) | Art) P(Nom | Adj(2)) P(Adv | Nom) P(SE | Adv)$ \\
$= 0,8 \cdot 0,5 \cdot 0,2 \cdot 0,9 \cdot 0,2 \cdot 0,8$ \\
$= 0,1152$ \\
\\
Das ist allerdings nur die Wahrscheinlichkeit für die passende Folge von Wortarten. Diese muss nun mit den Einzelwahrscheinlichkeiten für die Worte noch multipliziert werden. \\
$P($ \glqq He shoots the unwell well badly.\grqq $)$ \\
$= P(passende Wortartenfolge) \cdot P(He | Sub) \cdot P(shoots | Ver) \cdot P(the | Art) \cdot P(unwell | Adj(2)) \cdot P(well | Nom) \cdot P(badly | Adv)$ \\
$= 0,1152 \cdot 0,5 \cdot 0,5 \cdot 0,6 \cdot 0,5 \cdot 0,4 \cdot 0,5$
$= 0,0001728 = 0,01728 \%$ \\
\\
Die Wahrscheinlichkeit f\"ur den Satz \glqq He shoots the unwell well badly.\grqq betr\"agt 0,01728 \%.

\subsection*{2.c)}
$y_0 = She \Rightarrow x_0 \in \{Subjekt\}$ \\
$y_1 = is \Rightarrow x_1 \in \{Verb, Hilfsverb\}$ \\
$\Rightarrow x_2 \in \{Artikel, Adverb, Satzende, Adjektiv(1)\}$ \\
\\
$x_4 = Satzende \Rightarrow x_3 \in \{Verb, Adjektiv(1), Nomen, Adverb\}$ \\
$y_3 = well \Rightarrow x_3 \in \{Adjektiv(1), Nomen, Adverb, Adjektiv(2)\}$ \\
$\Rightarrow x_3 \in \{Adjektiv(1), Nomen, Adverb\}$ \\
$\Rightarrow x_2 \in \{Hilfsverb, Artikel, Adjektiv(2), Verb, Adjektiv(1), Nomen, Adverb\}$ \\

Aus diesen beiden Restriktionen f\"ur $x_2$ kann man folgern, dass: $x_2 \in \{Artikel, Adverb, Adjektiv(1)\}$



\pagebreak
\subsection*{2.d)}
\begin{figure}[!htbp]
\centering
\includegraphics[width = 1\columnwidth]{blatt6aufgabe2d}
\end{figure}

S\"atze mit genau 3 W\"ortern entsprechen Pfaden von der Wurzel (Subjekt) zu Blättern, die mit \glqq Satzende\grqq bezeichnet sind und diese Pfade sollen über 3 Kanten gehen. \\
Hierfür gibt es folgende geordnete Wortartkombinationen: \\
(i) Subjekt-Hilfsverb-Adjektiv(1)-Satzende
(ii) Subjekt-Verb-Adverb-Satzende

$P($ Satz mit genau 3 W\"ortern $) = P(x_0 =Sub,x_1=Hil,x_2=Adj(1),x_3=SE)+P(x_0=Sub,x_1=Ver,x_2=Adv,x_3=SE)$ \\
$= P(Hil | Sub) \cdot P(Adj(1) | Hil) \cdot P(SE | Adj(1))+P(Ver | Sub) \cdot P(Adv) | Ver) \cdot P(SE | Adv)$ \\
$=0,2 \cdot 1 \cdot 0,8 + 0,8 \cdot 0,2 \cdot 0,8$ \\
$= 0,16 + 0,128 = 0,288$ \\
\\
Die Wahrscheinlichkeit f\"ur einen Satz mit genau 3 W\"ortern betr\"agt 28,8 \%. \\
\\
$M((i))$ bezeichne die Menge der M\"oglichkeiten f\"ur den Fall (i) und $| M(i)|$ bezeichne die Anzahl an M\"oglichkeiten f\"ur den Fall (i), , f\"ur (ii) analog.\\
Mit $| Wortart |$ bezeichnen wir die Anzahl m\"oglicher Worte aus der Wortart. \\
$|M(i)| = |Subjekt| \cdot |Hilfsverb| \cdot |Adjektiv(1)| = 2^3 = 8$ \\
$|M(ii)| = |Subjekt| \cdot |Verb| \cdot |Adverb| = 2^3 = 8$ \\
Die Gesamtanzahl an M\"oglichkeit ergibt sich aus: $|M((i))| + |M((ii))|- |M((i)) \cap M((ii))|$
Dazu betrachte: $Subjekt \cap Subjekt = \{ He, She \}$ \\
$Hilfsverb \cap Verb = \{is\}$ \\
$Adjektiv(1) \cap Adverb = \{well\}$ \\
$\Rightarrow |M((i)) \cap M((ii))| = |Subjekt \cap Subjekt| \cdot |Hilfsverb \cap Verb| \cdot |Adjektiv(1) \cap Adverb| = 2$ \\
$\Rightarrow$ Anzahl verschiedener M\"oglichkeiten f\"ur einen Satz mit genau 3 W\"ortern $= 8+8-2 = 14$

\end{document}
