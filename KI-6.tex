\documentclass[a4paper]{article}

\usepackage[ngerman]{babel}
\usepackage[utf8]{inputenc}
\usepackage{amsmath}
\usepackage{amssymb}
\usepackage{amsthm}
\usepackage{booktabs}
\usepackage{array}
\usepackage{tabularx}
\usepackage{color}
\usepackage{ulem}
\usepackage{fancyhdr}
\usepackage{graphicx}
\usepackage{geometry}
\usepackage{polynom}
\usepackage{pgf,tikz}
\usepackage{amsfonts}
\usepackage{cancel}
\usepackage{mathcomp}
\usepackage{mathrsfs}
\usepackage{multirow}
\usepackage{dsfont}
\usepackage{eurosym}
\usepackage{
    amscd,
    amsfonts,
    amsmath,
    amssymb,
    amsthm,
}
\usepackage{tikz}
\usepackage{stmaryrd}
\usepackage{ulsy}


\begin{document}

  \begin{flushright}
    \today
  \end{flushright}
  \begin{center}
    \Large\textbf{{GKI - Hausaufgaben 5}}\\
  \end{center}

  \begin{center}
        \large\textsl{Tao Xu, 343390 - Mitja Richter, 324680 - Björn Kapelle, 320438 - Marcus Weber, 320402}\\
  \end{center}
  
\section*{Aufgabe 1}

\subsection*{1.a)}

\subsection*{1.b)}

\section*{Aufgabe 2}

\subsection*{2.a)}
\begin{figure}[!htbp]
\centering
\includegraphics[width = 1\columnwidth]{blatt6aufgabe2a}
\end{figure}

\subsection*{2.b)}
\glqq He\grqq $\in$ \{Subjekt\} \\
\glqq shoots\grqq $\in$ \{Verb\} \\
\glqq the\grqq $\in$ \{Artikel\} \\
\glqq unwell\grqq $\in$ \{Adjektiv(1), Adjektiv(2)\} \\
\glqq well\grqq $\in$ \{Adjektiv(1), Nomen, Adverb, Adjektiv(2)\} \\
\glqq badly\grqq $\in$ \{Adverb\} \\
\pagebreak
Daraus folgt das folgende Bild.
\begin{figure}[!htbp]
\centering
\includegraphics[width = 1\columnwidth]{blatt6aufgabe2b}
\end{figure}

Die einzigen möglichen Folgen von Wortarten beschreiben Pfade von der Wurzel (Subjekt) zum Blatt (Satzende) bei dem jede Kante eine positive Wahrscheinlichkeit hat.
Es gibt nur einen Pfad, der das erf\"ullt: Subjekt-Verb-Artikel-Adjektiv(2)-Nomen-Adverb-Satzende. \\
Insbesondere gibt es damit eine M\"oglichkeit diesen Satz zu bilden. \\
Im weiteren verwenden wir Abkürzungen:  \\
Sub - Subjekt \\
Ver - Verb \\
Hil - Hilfsverb \\
Art - Artikel \\
Adj(2) - Adjektiv(2) \\
Nom - Nomen \\
Adv - Adverb \\
SE - Satzende \\
\\
$P($passende Wortartenfolge$)$
$= P(x_0=Sub, x_1=Ver, x_2= Art, x_3=Adj(2), x_4=Nom, x_5=Adv, x_6=SE)$ \\
$= P(Ver | Sub)  P(Art | Ver) P(Adj(2) | Art) P(Nom | Adj(2)) P(Adv | Nom) P(SE | Adv)$ \\
$= 0,8 \cdot 0,5 \cdot 0,2 \cdot 0,9 \cdot 0,2 \cdot 0,8$ \\
$= 0,1152$ \\
\\
Das ist allerdings nur die Wahrscheinlichkeit für die passende Folge von Wortarten. Diese muss nun mit den Einzelwahrscheinlichkeiten für die Worte noch multipliziert werden. \\
$P($ \glqq He shoots the unwell well badly.\grqq $)$ \\
$= P(passende Wortartenfolge) \cdot P(He | Sub) \cdot P(shoots | Ver) \cdot P(the | Art) \cdot P(unwell | Adj(2)) \cdot P(well | Nom) \cdot P(badly | Adv)$ \\
$= 0,1152 \cdot 0,5 \cdot 0,5 \cdot 0,6 \cdot 0,5 \cdot 0,4 \cdot 0,5$
$= 0,0001728 = 0,01728 \%$ \\
\\
Die Wahrscheinlichkeit f\"ur den Satz \glqq He shoots the unwell well badly.\grqq betr\"agt 0,01728 \%.

\subsection*{2.c)}
$y_0 = She \Rightarrow x_0 \in \{Subjekt\}$ \\
$y_1 = is \Rightarrow x_1 \in \{Verb, Hilfsverb\}$ \\
$\Rightarrow x_2 \in \{Artikel, Adverb, Satzende, Adjektiv(1)\}$ \\
\\
$x_4 = Satzende \Rightarrow x_3 \in \{Verb, Adjektiv(1), Nomen, Adverb\}$ \\
$y_3 = well \Rightarrow x_3 \in \{Adjektiv(1), Nomen, Adverb, Adjektiv(2)\}$ \\
$\Rightarrow x_3 \in \{Adjektiv(1), Nomen, Adverb\}$ \\
$\Rightarrow x_2 \in \{Hilfsverb, Artikel, Adjektiv(2), Verb, Adjektiv(1), Nomen, Adverb\}$ \\

Aus diesen beiden Restriktionen f\"ur $x_2$ kann man folgern, dass: $x_2 \in \{Artikel, Adverb, Adjektiv(1)\}$



\pagebreak
\subsection*{2.d)}
\begin{figure}[!htbp]
\centering
\includegraphics[width = 1\columnwidth]{blatt6aufgabe2d}
\end{figure}

S\"atze mit genau 3 W\"ortern entsprechen Pfaden von der Wurzel (Subjekt) zu Blättern, die mit \glqq Satzende\grqq bezeichnet sind und diese Pfade sollen über 3 Kanten gehen. \\
Hierfür gibt es folgende geordnete Wortartkombinationen: \\
(i) Subjekt-Hilfsverb-Adjektiv(1)-Satzende
(ii) Subjekt-Verb-Adverb-Satzende

$P($ Satz mit genau 3 W\"ortern $) = P(x_0 =Sub,x_1=Hil,x_2=Adj(1),x_3=SE)+P(x_0=Sub,x_1=Ver,x_2=Adv,x_3=SE)$ \\
$= P(Hil | Sub) \cdot P(Adj(1) | Hil) \cdot P(SE | Adj(1))+P(Ver | Sub) \cdot P(Adv) | Ver) \cdot P(SE | Adv)$ \\
$=0,2 \cdot 1 \cdot 0,8 + 0,8 \cdot 0,2 \cdot 0,8$ \\
$= 0,16 + 0,128 = 0,288$ \\
\\
Die Wahrscheinlichkeit f\"ur einen Satz mit genau 3 W\"ortern betr\"agt 28,8 \%. \\
\\
$M((i))$ bezeichne die Menge der M\"oglichkeiten f\"ur den Fall (i) und $| M(i)|$ bezeichne die Anzahl an M\"oglichkeiten f\"ur den Fall (i), , f\"ur (ii) analog.\\
Mit $| Wortart |$ bezeichnen wir die Anzahl m\"oglicher Worte aus der Wortart. \\
$|M(i)| = |Subjekt| \cdot |Hilfsverb| \cdot |Adjektiv(1)| = 2^3 = 8$ \\
$|M(ii)| = |Subjekt| \cdot |Verb| \cdot |Adverb| = 2^3 = 8$ \\
Die Gesamtanzahl an M\"oglichkeit ergibt sich aus: $|M((i))| + |M((ii))|- |M((i)) \cap M((ii))|$
Dazu betrachte: $Subjekt \cap Subjekt = \{ He, She \}$ \\
$Hilfsverb \cap Verb = \{is\}$ \\
$Adjektiv(1) \cap Adverb = \{well\}$ \\
$\Rightarrow |M((i)) \cap M((ii))| = |Subjekt \cap Subjekt| \cdot |Hilfsverb \cap Verb| \cdot |Adjektiv(1) \cap Adverb| = 2$ \\
$\Rightarrow$ Anzahl verschiedener M\"oglichkeiten f\"ur einen Satz mit genau 3 W\"ortern $= 8+8-2 = 14$

\end{document}